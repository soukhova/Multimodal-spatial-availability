% Template for PLoS
% Version 3.5 March 2018
%
% % % % % % % % % % % % % % % % % % % % % %
%
% -- IMPORTANT NOTE
%
% This template contains comments intended
% to minimize problems and delays during our production
% process. Please follow the template instructions
% whenever possible.
%
% % % % % % % % % % % % % % % % % % % % % % %
%
% Once your paper is accepted for publication,
% PLEASE REMOVE ALL TRACKED CHANGES in this file
% and leave only the final text of your manuscript.
% PLOS recommends the use of latexdiff to track changes during review, as this will help to maintain a clean tex file.
% Visit https://www.ctan.org/pkg/latexdiff?lang=en for info or contact us at latex@plos.org.
%
%
% There are no restrictions on package use within the LaTeX files except that
% no packages listed in the template may be deleted.
%
% Please do not include colors or graphics in the text.
%
% The manuscript LaTeX source should be contained within a single file (do not use \input, \externaldocument, or similar commands).
%
% % % % % % % % % % % % % % % % % % % % % % %
%
% -- FIGURES AND TABLES
%
% Please include tables/figure captions directly after the paragraph where they are first cited in the text.
%
% DO NOT INCLUDE GRAPHICS IN YOUR MANUSCRIPT
% - Figures should be uploaded separately from your manuscript file.
% - Figures generated using LaTeX should be extracted and removed from the PDF before submission.
% - Figures containing multiple panels/subfigures must be combined into one image file before submission.
% For figure citations, please use "Fig" instead of "Figure".
% See http://journals.plos.org/plosone/s/figures for PLOS figure guidelines.
%
% Tables should be cell-based and may not contain:
% - spacing/line breaks within cells to alter layout or alignment
% - do not nest tabular environments (no tabular environments within tabular environments)
% - no graphics or colored text (cell background color/shading OK)
% See http://journals.plos.org/plosone/s/tables for table guidelines.
%
% For tables that exceed the width of the text column, use the adjustwidth environment as illustrated in the example table in text below.
%
% % % % % % % % % % % % % % % % % % % % % % % %
%
% -- EQUATIONS, MATH SYMBOLS, SUBSCRIPTS, AND SUPERSCRIPTS
%
% IMPORTANT
% Below are a few tips to help format your equations and other special characters according to our specifications. For more tips to help reduce the possibility of formatting errors during conversion, please see our LaTeX guidelines at http://journals.plos.org/plosone/s/latex
%
% For inline equations, please be sure to include all portions of an equation in the math environment.
%
% Do not include text that is not math in the math environment.
%
% Please add line breaks to long display equations when possible in order to fit size of the column.
%
% For inline equations, please do not include punctuation (commas, etc) within the math environment unless this is part of the equation.
%
% When adding superscript or subscripts outside of brackets/braces, please group using {}.
%
% Do not use \cal for caligraphic font.  Instead, use \mathcal{}
%
% % % % % % % % % % % % % % % % % % % % % % % %
%
% Please contact latex@plos.org with any questions.
%
% % % % % % % % % % % % % % % % % % % % % % % %

\documentclass[10pt,letterpaper]{article}
\usepackage[top=0.85in,left=2.75in,footskip=0.75in]{geometry}

% amsmath and amssymb packages, useful for mathematical formulas and symbols
\usepackage{amsmath,amssymb}

% Use adjustwidth environment to exceed column width (see example table in text)
\usepackage{changepage}


% textcomp package and marvosym package for additional characters
\usepackage{textcomp,marvosym}

% cite package, to clean up citations in the main text. Do not remove.
% \usepackage{cite}

% Use nameref to cite supporting information files (see Supporting Information section for more info)
\usepackage{nameref,hyperref}

% line numbers
\usepackage[right]{lineno}

% ligatures disabled
\usepackage{microtype}
\DisableLigatures[f]{encoding = *, family = * }

% color can be used to apply background shading to table cells only
\usepackage[table]{xcolor}

% array package and thick rules for tables
\usepackage{array}

% create "+" rule type for thick vertical lines
\newcolumntype{+}{!{\vrule width 2pt}}

% create \thickcline for thick horizontal lines of variable length
\newlength\savedwidth
\newcommand\thickcline[1]{%
  \noalign{\global\savedwidth\arrayrulewidth\global\arrayrulewidth 2pt}%
  \cline{#1}%
  \noalign{\vskip\arrayrulewidth}%
  \noalign{\global\arrayrulewidth\savedwidth}%
}

% \thickhline command for thick horizontal lines that span the table
\newcommand\thickhline{\noalign{\global\savedwidth\arrayrulewidth\global\arrayrulewidth 2pt}%
\hline
\noalign{\global\arrayrulewidth\savedwidth}}


% Remove comment for double spacing
%\usepackage{setspace}
%\doublespacing

% Text layout
\raggedright
\setlength{\parindent}{0.5cm}
\textwidth 5.25in
\textheight 8.75in

% Bold the 'Figure #' in the caption and separate it from the title/caption with a period
% Captions will be left justified
\usepackage[aboveskip=1pt,labelfont=bf,labelsep=period,justification=raggedright,singlelinecheck=off]{caption}
\renewcommand{\figurename}{Fig}

% Use the PLoS provided BiBTeX style
% \bibliographystyle{plos2015}

% Remove brackets from numbering in List of References
\makeatletter
\renewcommand{\@biblabel}[1]{\quad#1.}
\makeatother



% Header and Footer with logo
\usepackage{lastpage,fancyhdr,graphicx}
\usepackage{epstopdf}
%\pagestyle{myheadings}
\pagestyle{fancy}
\fancyhf{}
%\setlength{\headheight}{27.023pt}
%\lhead{\includegraphics[width=2.0in]{PLOS-submission.eps}}
\rfoot{\thepage/\pageref{LastPage}}
\renewcommand{\headrulewidth}{0pt}
\renewcommand{\footrule}{\hrule height 2pt \vspace{2mm}}
\fancyheadoffset[L]{2.25in}
\fancyfootoffset[L]{2.25in}
\lfoot{\today}

%% Include all macros below

\newcommand{\lorem}{{\bf LOREM}}
\newcommand{\ipsum}{{\bf IPSUM}}


% tightlist command for lists without linebreak
\providecommand{\tightlist}{%
  \setlength{\itemsep}{0pt}\setlength{\parskip}{0pt}}


% Pandoc citation processing
\newlength{\cslhangindent}
\setlength{\cslhangindent}{1.5em}
\newlength{\csllabelwidth}
\setlength{\csllabelwidth}{3em}
\newlength{\cslentryspacingunit} % times entry-spacing
\setlength{\cslentryspacingunit}{\parskip}
% for Pandoc 2.8 to 2.10.1
\newenvironment{cslreferences}%
  {}%
  {\par}
% For Pandoc 2.11+
\newenvironment{CSLReferences}[2] % #1 hanging-ident, #2 entry spacing
 {% don't indent paragraphs
  \setlength{\parindent}{0pt}
  % turn on hanging indent if param 1 is 1
  \ifodd #1
  \let\oldpar\par
  \def\par{\hangindent=\cslhangindent\oldpar}
  \fi
  % set entry spacing
  \setlength{\parskip}{#2\cslentryspacingunit}
 }%
 {}
\usepackage{calc}
\newcommand{\CSLBlock}[1]{#1\hfill\break}
\newcommand{\CSLLeftMargin}[1]{\parbox[t]{\csllabelwidth}{#1}}
\newcommand{\CSLRightInline}[1]{\parbox[t]{\linewidth - \csllabelwidth}{#1}\break}
\newcommand{\CSLIndent}[1]{\hspace{\cslhangindent}#1}

\usepackage{multirow}
\usepackage{multicol}
\usepackage{colortbl}
\usepackage{hhline}
\newlength\Oldarrayrulewidth
\newlength\Oldtabcolsep
\usepackage{longtable}
\usepackage{array}
\usepackage{hyperref}
\usepackage{float}
\usepackage{wrapfig}


\usepackage{forarray}
\usepackage{xstring}
\newcommand{\getIndex}[2]{
  \ForEach{,}{\IfEq{#1}{\thislevelitem}{\number\thislevelcount\ExitForEach}{}}{#2}
}

\setcounter{secnumdepth}{0}

\newcommand{\getAff}[1]{
  \getIndex{#1}{}
}

\begin{document}
\vspace*{0.2in}


% Title must be 250 characters or less.
\begin{flushleft}
{\Large
\textbf\newline{Multimodal Spatial Availability: A Singly-Constrained
Measure of Competitive Accessibility by Multiple
Modes} % Please use "sentence case" for title and headings (capitalize only the first word in a title (or heading), the first word in a subtitle (or subheading), and any proper nouns).
}
\newline
% Insert author names, affiliations and corresponding author email (do not include titles, positions, or degrees).
\\
Anastasia Soukhov\textsuperscript{\getAff{Department of Earth,
Environment and Society, McMaster University, Canada}},
Javier Tarriño-Ortiz\textsuperscript{\getAff{Universidad Politécnica de
Madrid, Spain}},
Julio A. Soria-Lara\textsuperscript{\getAff{Urban and Regional Planning,
Universidad Politécnica de Madrid, Spain}},
Antonio Páez\textsuperscript{\getAff{Department of Earth, Environment
and Society, McMaster University, Canada}}\\
\bigskip
\bigskip
\end{flushleft}
% Please keep the abstract below 300 words
\section*{Abstract}
An increasing number of studies within the domain of transportation
planning are concerned with the inequities in accessibility to
opportunities. A dimension of these inequities arises from differences
in access by mode type (e.g., commuting using a car as opposed to
transit). However, methods implemented in current accessibility
literature are lacking within the context of multimodal analysis. This
paper presents an extension of spatial availability, a
singly-constrained competitive accessibility measure, for the context of
multimodal accessibility analysis. We first illustrate the features of
spatial availability that lend itself to multimodal analysis. We then
demonstrate its use on the case study of Low Emission Zones in Madrid
(Spain) and highlight how this policy intervention changes the
accessibility of populations using different modes. In summary, spatial
availability can be used to create and interpret multimodal policy
intervention scenarios unlike previous methods: this creation and
interpretation can help regions envision a more sustainable and
equitable access-to-opportunity landscape.

% Please keep the Author Summary between 150 and 200 words
% Use first person. PLOS ONE authors please skip this step.
% Author Summary not valid for PLOS ONE submissions.

\linenumbers

% Use "Eq" instead of "Equation" for equation citations.
\hypertarget{abstract}{%
\section{Abstract}\label{abstract}}

An increasing number of studies within the domain of transport are
concerned with the inequities in accessibility to opportunities. A
dimension of these inequities arise from differences in access by mode
type (e.g., the number of work opportunities that can be reached using a
car as opposed to transit in a city). However, methods assessing
multimodal accessibility in the literature fall short as aspects of
competition for opportunities and the explicit methodological
acknowledgement of opportunities being \emph{finite} are lacking. In
this vein, this paper presents an extension of \emph{spatial
availability}, a singly-constrained competitive accessibility measure,
for the context of multimodal accessibility analysis. We first
illustrate the features of spatial availability that lends itself to
multimodal analysis. We then demonstrate its use on the case study of
Low Emission Zones in Madrid (Spain) and highlight how this policy
intervention changes the accessibility for locations where populations
use different modes. In summary, spatial availability can be used to
create and interpret multimodal policy intervention scenarios unlike
previous methods: this creation and interpretation can help regions
envision a more sustainable and equitable access-to-opportunity
landscape by better identifying differences in accessibility afforded by
different modes.

\hypertarget{introduction}{%
\section{Introduction}\label{introduction}}

Implementing urban policies that re-shape cities through accessibility
gains (i.e., the \emph{potential to interact} with opportunities as a
result of land-use mix and transport systems as originally defined by
{[}1{]}) have been widely applied within the transportation literature
and is increasingly discussed by planners {[}2--5{]}. An important
challenge in the identification of interventions that equitably
transform cities is the effective evaluation of \emph{trade-offs}:
cities are complex and dynamic ecologies, and advantaging one component
of the city can disadvantage another area, population, or sub-component
In this way, policy evaluation should take a \emph{systems} approach
{[}6{]}. One way of considering systems is from the perspective of the
\emph{finite}. As an illustration, consider the amount of transport
space within a city: the amount is typically finite so re-allocating
road space away from one mode directly impacts the performance of the
others (see the literature on road space reallocation e.g., {[}7{]}).
Evaluating policy impacts in the context of \emph{finity} provides a way
to contextualize the balance of trade-offs that the citizens of a city
should tolerate.

From the perspective of urban transport systems, location-based
accessibility measures have been used in the context of policy
evaluation. For instance, {[}8{]} assesses the transit accessibility
gains to healthcare and employment opportunities for disadvantaged
neighbourhood in Columbus, Ohio, USA after the transit system's
re-design and introduction of a rapid bus system. However, a limitation
of this study, like others that implement accessibility measures, is
they do not calculate results under a \emph{constrained} framework i.e.,
one of \emph{finity}. The citizens of Columbus should experience
quantitative accessibility gains - but is it at the expense of access
using other modes? As another example, {[}9{]} implements a modified
cumulative opportunity measure to assess differences between private
vehicle and transit system accessibility to jobs in Melbourne, but a
similar question remains: does the accessibility afforded to the private
vehicle using population come at the expense of accessibility losses to
transit users?

{[}8{]} and {[}9{]} both use \emph{non-competitive} accessibility
measures. There is a branch of location-based accessibility measures
that do incorporate the effect of competition for opportunities by the
population in the region. However, we argue that these existing methods
fall short in acknowledging the \emph{finity} of opportunities. For
instance, {[}10{]} applies a competitive measure, two-step floating
catchment approach (2SFCA), for the case of access to healthcare
services in Florida for both a multimodal network and a single modal
network. While the differences in modal access are discussed, the
question of how the advantage in access afforded by one mode over
another impacts access for different mode users is unanswered.

This question of how much one mode-using population can access at the
expense of another mode-using population is a pertinent equity question
in the evaluation of policy scenarios that are multimodal. For instance,
consider the impact of a low emission zone (LEZ). LEZ is a policy of
spatial and modal discrimination: the circulation of vehicles that are
excessively polluting are restricted in specific regions. In the
recognition that opportunities are finite, the implementation of a LEZ
explicitly reduces the access that the population using polluting
vehicles has to opportunities. This restriction allows the population
using other more sustainable modes to potentially have a higher level of
access than before the LEZ implementation. This evaluation is especially
urgent as LEZ are currently in effect in cities globally; their
reception has been mixed {[}11{]} and may be having negative impacts on
disadvantaged populations who have become mobility-restricted
{[}12,13{]}. Measures that evaluate the accessibility of modes given
both \emph{constrained} and \emph{competitive} considerations are
lacking in the literature, but are needed, to evaluate such policy
interventions impact on accessibility.

In {[}14{]}, we introduce spatial availability, a type of location-based
accessibility measure that is both \emph{constrained} and
\emph{competitive}. In this paper, we extend the spatial availability
measure into a multimodal framework and explore its use in answering the
question outlined: \emph{``given opportunities are finite, how many are
available to a given location depending on the mode used?''.} The answer
to this question quantifies how many opportunities can be accessed,
considering competition, for different modes. To foreground this
exploration, in Section 2, we discuss short falls of a few existing
location-based measures in comparison to spatial availability through a
synthetic example. In Section 3, the spatial availability of an
empirical example of the LEZ in the city of Madrid, Spain is calculated.
We demonstrate how the restriction of car circulation could have
impacted the spatial availability of opportunities for each
sub-population using transit, cycling and walking modes. In Section 4,
we provide concluding remarks on the strengths of the use of spatial
availability as a multimodal accessibility measure and potential future
uses in policy planning scenarios.

\hypertarget{a-review-of-multimodal-accessibility-measures}{%
\section{A review of multimodal accessibility
measures}\label{a-review-of-multimodal-accessibility-measures}}

Location-based accessibility indicators are quantitative measures of
\emph{potential} interaction with opportunities for locations within a
given region: they are a product of the relationship between land-use
and transport systems. Arguably the most commonly used location-based
measured are cumulative opportunity measures and weighted cumulative
opportunity measures {[}2{]}. These measures weight the opportunities
that can be potentially interacted with from origin \(i\) to destination
\(j\) based on some sort of travel cost function (e.g., travel time,
fare, travel distance) otherwise known as a travel impedance function
\(f^{m}(c^m_{ij})\). Many weighted cumulative opportunities (often
referred to as the gravity-based measure) originate from the measure
proposed by {[}1{]}, which can take the following multimodal form:
\(S_i^m = \sum_j O_j f^m(c_{ij}^m)\) where \(m\) is a set of modes which
have mode-specific travel costs \((c_{ij}^m)\) and travel impedance
functions \(f^m(\cdot)\).

The Hansen-type measure does not consider competition between modes nor
is it constrained. As an example, the work of {[}15{]} uses the
Hansen-type measure to measure the potential interaction with retail
locations using walking, public transit, and car modes \(m\). \(S_i^m\)
is the sum of retail locations \(j\) that can potentially be interacted
with under the travel impedance as calculated for each \(i\) and \(m\).
In other words, each \(i\) has three \(S_i\) values, one per \(m\). In
this work, they demonstrate that the car mode has the highest
\(S_i^{m=car}\) values in the majority of \(i\), i.e., populations using
a car can potentially interact with the most retail opportunities than
populations using other modes. However, the higher \(S_i^{m=car}\)
values are not a result of lower \(S_i^{m}\) values for other modes: it
is not assumed that car-using populations potentially accessing more
opportunities take away potential opportunities for other populations
within the measure (no consideration for competition). This measure is
also not constrained: there is no global maximum for \(S_i\) or
\(S_i^m\) values, they are presented as a population normalized
accessibility index. This makes the interpretation of the `potentially
interacted opportunities' relative to the region, making comparisons of
the results across different regions challenging.

However, opportunities in a region can be considered finite. There are
only so many school-seats, hospital capacity, jobs, etc., in a region
and if one person interacts with an opportunity at a given time, it is
taken. As such, if one person is advantaged and has the ability to reach
more opportunities through a lower travel-cost mode, than they have more
opportunities to potentially interact with more opportunities than other
people. From the other perspective, their are fewer opportunities left
to be potentially interacted with for populations using higher
travel-cost modes. In this way, populations using modes with a higher
travel impedance are at a higher access disadvantage than populations
using lower travel impedance modes. This recognition is the motivation
behind integrating \emph{competition} for opportunities within
multimodal accessibility measures. Arguably one of the most popular
competitive location-based accessibility measures is the two-step
floating catchment area (2SFCA) approach popularized by {[}16{]} who
simplified the approach proposed by {[}17{]} (with similar
considerations for competition in {[}18{]} and {[}19{]}).

The Shen-type accessibility measure's formulation is:
\(a_i^m = \sum_j \frac{O_jf^m(c_{ij}^m)}{\sum_m D_j^m}\) where \(D_j^m\)
is the potential demand for opportunities equal to travel impedance
weighted population \(\sum_i P_i^m f^m(c_{ij}^m)\). In this way, the
Shen-type measure can be understood as a ratio of the potential
opportunity supply over the potential demand for opportunities. The
measure considers competition, but it is \emph{non-constrained}. A score
of competitive potential accessibility associated is associated with
each location \(i\) for each mode \(m\), but there are no global
maximums. In other words, it is difficult to interpret the meaning of
differences in Shen-type accessibility scores between modes.

To illustrate, {[}20{]} calculates \(a_i^m\) to jobs for different
income-group populations in Shenzhen (China) using
\(m = \text{public transit}\) and \(m={car}\). They demonstrate that
\(i\) with low-income populations have lower \(a_i^m\) than \(i\) with
higher-income populations. Further, they demonstrate that
\(a_i^{m=\text{public transit}}\) is lower than \(a_i^{m=\text{car}}\)
at many \(i\), arguing that this may put \(i\) with lower-income
populations in a further disadvantage. \(a_i\) and/or \(a_i^m\) are used
to compare relative spatial differences in overall competitive
accessibility and modal competitive accessibility, but because there is
no global maximum, making it is difficult to interpret the significance
between differences in \(a_i^{m}\) values. Questions such as: what is
the impact that competition has on the difference in \(a_i^m\) values?
How does impact vary spatially? And what is the interpretation of this
difference? are left unanswered.

Spatial availability improves on previous multimodal accessibility
approaches as it considers \emph{competition} in the potential
interaction with opportunities in a \emph{constrained} framework (e.g.,
finite opportunities). This is done by considering: 1) competition
between mass effect (e.g., the advantage of sub-populations residing in
relatively low population-density and high opportunity-proximate areas)
and 2) competition between travel impedance (e.g., sub-populations with
relatively low travel-impedance) through a proportional allocation
mechanism. The following sub-section demonstrates how spatial
availability compares to the Hansen-type and Shen-type measures through
a synthetic example.

\hypertarget{multimodal-spatial-availability-v_im}{%
\subsection{\texorpdfstring{Multimodal spatial availability
\(V_i^m\)}{Multimodal spatial availability V\_i\^{}m}}\label{multimodal-spatial-availability-v_im}}

In brief, we define the \emph{spatial availability} at \(i\) (\(V_{i}\))
as the proportion of all opportunities in the region \(O\) that are
allocated to location \(i\) from all opportunity destinations \(j\).
\(V_{i}\) is a value of how many opportunities are available to each
location \(i\) out of all the opportunities in the region. The general
formulation of spatial availability \(V_{i}\) is shown in Equation
(\ref{eq:spatial-availability-general}):

\begin{equation}
\label{eq:spatial-availability-general}
V_i = \sum_{j=1}^J O_jF^t_{ij}
\end{equation}

\noindent Where:

\begin{itemize}
\tightlist
\item
  \(F^t_{ij}\) is a balancing factor that depends on the demand for
  opportunities \(O_j\) and cost of movement in the system
  \(f(c_{ij})\).
\item
  \(V_i\) is the number of spatially available opportunities at \(i\);
  the sum of \(V_{i}\) is equivalent to the total sum of opportunities
  in the region (i.e., \(\sum_j O_j = \sum_i V_i\))
\end{itemize}

The spatial availability measure is introduced in {[}14{]}. Spatial
availability's unique feature is the balancing factor \(F^t_{ij}\), a
proportional allocation mechanisms, that ensures the \(V_i\) calculated
for each \(i\) sums, across all \(i\) in the region, to equal the total
number of opportunities in the region. As such, spatial availability is
a \emph{competitive} and \emph{constrained} accessibility measure as
\(F^t_{ij}\) handles the number of opportunities in the region in a
finite way (proportional allocation). \(F^t_{ij}\) consists of two
components: a population-based balancing factor
\(F^p_{i} = \frac{P_i}{\sum_i P_i}\) and an impedance-based balancing
factor \(F^c_{ij} = \frac{F^c_{ij}}{\sum_j F^c_{ij}}\) that,
respectively, allocate opportunities to \(i\) in proportion to the size
of the population at \(i\) (the mass effect) and the cost of reaching
opportunities at \(j\) (the impedance effect).

\(F^p_{i}\) and \$ F\^{}c\_\{ij\}\$ are calculated for each \(i\) such
that they both equal 1 when summed across all \(i\) in the region (e.g.,
\(\sum_i F^p_{i} = 1\) and \(\sum_i F^c_{ij} = 1\)). These balancing
factors are combined multiplicatively to yield \(F^t_{ij}\) which
ensures that a proportion of the opportunities \(O_j\) are allocated to
each \(i\) accordingly. In other words, assuming a finite number of
opportunities in the region, \(F^t_{ij}\) proportionally allocates
\(O_j\) to each \(i\) such that the resulting \(V_i\) value represents
the number of opportunities \emph{spatially available} to the population
at \(i\). This value can be seen to represent spatial availability as it
is a proportion of the opportunities in the region (i.e.,
\(\sum_j O_j = \sum_i V_i\)).

The focus of this paper is to extend \(V_i\) for multimodal
applications. To do so, the balancing factors are reformulated to yield
a proportional value for the set of modes \(m\) used by populations at
each \(i\). As these factors are proportional, \(F^{pm}_{i}\) and
\(F^{cm}_{ij}\) can be summed up across each \(m\) at each \(i\) and
across all \(i\) to equal to 1. They are also similarly combined
multiplicatively to obtain their joint effect, represented as the
combined balancing factor \(F^{tm}_{ij}\) detailed in Equation
(\ref{eq:multimodal-balancing-factors}).

\begin{equation}
\label{eq:multimodal-balancing-factors}
F^{tm}_{ij} = \frac{F^{pm}_{i} \cdot F^{cm}_{ij}}{\sum_{m=1}^M \sum_{i=1}^N F^{pm}_{i} \cdot F^{cm}_{ij}}
\end{equation}

\noindent Where:

\begin{itemize}
\tightlist
\item
  The population balancing factor for each \(m\) at each \(i\) is
  \(F^{pm}_{i} = \frac{P_{i}^m}{\sum_{m}\sum_{i} P_{i}^m}\)
\item
  The cost of travel balancing factor for each \(m\) at \(i\) is
  \(F_{ij}^{cm} = \frac{f(c_{ij}^m)}{\sum_{m} \sum_{i} f(c_{ij}^m)}\)
\end{itemize}

Implementing \(F^{tm}_{ij}\), the following Equation
(\ref{eq:spatial-availability-multimodal}) demonstrates the multimodal
configuration of spatial availability \(V_i^m\):

\begin{equation}
\label{eq:spatial-availability-multimodal}
V^m_{i} = \sum_{j=1}^J O_j\ F^{tm}_{ij}
\end{equation}

\noindent Where:

\begin{itemize}
\tightlist
\item
  \(m\) is a set of modes used by populations in the region.
\item
  \(F^{tm}_{ij}\) is a balancing factor \(F^t_{ij}\) for each \(m\) at
  each \(i\).
\item
  \(V^m_{i}\) is the spatial availability \(V_{i}\) for mode \(m\) at
  each \(i\); the sum of \(V^m_{i}\) for all \(m\) at each \(i\) is
  equivalent to the total sum of opportunities in the region (i.e.,
  \(\sum_j O_j = \sum_i V_i = \sum_{m} \sum_{i} V^m_{i}\))
\end{itemize}

\hypertarget{synthetic-example}{%
\subsection{Synthetic example}\label{synthetic-example}}

Consider the following: Figure \ref{fig:Fig1} depicts a region with
population and jobs at three population centers (\(A\), \(B\), \(C\))
and three employment centers (\(1\), \(2\), \(3\)). The population at
each population center is divided into two sub-populations, one using a
faster mode \(z\) and another using a slower mode \(x\), to travel to
employment centers. Population center \(A\) is Suburban: it is closest
to its own relatively large employment center at \(1\), close to the
Urban's equally large employment center \(2\), and has a population that
is smaller than the Urban \(B\) and larger than the Satellite \(C\).
\(B\) has the largest \(x\)-using population, followed by then \(A\),
then \(C\). This synthetic example was inspired by the single-mode
example used in {[}17{]} and reconfigured in {[}14{]}.

\begin{figure}

{\centering \includegraphics[width=1\linewidth]{images/Fig1} 

}

\caption{\label{fig:Fig1} Multimodal synthetic example: locations of employment centers (in orange), population centers (in blue), number of jobs and population, and travel times for two modes (slower mode x and faster mode z).}\label{fig:synthetic-example-plot}
\end{figure}

From the perspective of access to a \emph{finite} amount of
opportunities in the region (\(210,000\) jobs), the sub-population that
is most proximate to jobs, furthest from densely populated centers, and
is using the lowest travel-cost mode \(z\) can potentially access the
most job opportunities. This appears to be the sub-population at \(A\)
using \(z\). From the other perspective, sub-populations located in
opposite conditions (i.e., further away from jobs, close to dense
populations, and using \(x\)) are at a relative job opportunity access
\emph{disadvantage}. From the perspective of inequities, the competition
for opportunities between different mode-using populations matters as it
reflects how well the land-use and transport system serves (or doesn't
serve) them.

\global\setlength{\Oldarrayrulewidth}{\arrayrulewidth}

\global\setlength{\Oldtabcolsep}{\tabcolsep}

\setlength{\tabcolsep}{0pt}

\renewcommand*{\arraystretch}{1.5}



\providecommand{\ascline}[3]{\noalign{\global\arrayrulewidth #1}\arrayrulecolor[HTML]{#2}\cline{#3}}

\begin{longtable}[c]{|p{0.88in}|p{0.41in}|p{1.05in}|p{0.58in}|p{1.05in}|p{0.58in}|p{1.05in}}

\caption{Accessibility\ values\ at\ each\ origin\ per\ mode\ m\ at\ each\ origin\ i\ and\ aggregated\ between\ modes\ for\ each\ i\ for\ the\ synthetic\ example.}\\

\ascline{1.5pt}{666666}{1-7}

\multicolumn{1}{>{\raggedright}m{\dimexpr 0.88in+0\tabcolsep}}{\textcolor[HTML]{000000}{\fontsize{11}{11}\selectfont{i}}} & \multicolumn{1}{>{\raggedright}m{\dimexpr 0.41in+0\tabcolsep}}{\textcolor[HTML]{000000}{\fontsize{11}{11}\selectfont{m}}} & \multicolumn{1}{!{\color[HTML]{666666}\vrule width 1pt}>{\raggedleft}m{\dimexpr 1.05in+0\tabcolsep}}{\textcolor[HTML]{000000}{\fontsize{11}{11}\selectfont{S}}\textcolor[HTML]{000000}{\textsubscript{\fontsize{11}{11}\selectfont{i}}}\textcolor[HTML]{000000}{\textsuperscript{\fontsize{11}{11}\selectfont{m}}}} & \multicolumn{1}{>{\raggedleft}m{\dimexpr 0.58in+0\tabcolsep}}{\textcolor[HTML]{000000}{\fontsize{11}{11}\selectfont{a}}\textcolor[HTML]{000000}{\textsubscript{\fontsize{11}{11}\selectfont{i}}}\textcolor[HTML]{000000}{\textsuperscript{\fontsize{11}{11}\selectfont{m}}}} & \multicolumn{1}{>{\raggedleft}m{\dimexpr 1.05in+0\tabcolsep}}{\textcolor[HTML]{000000}{\fontsize{11}{11}\selectfont{V}}\textcolor[HTML]{000000}{\textsubscript{\fontsize{11}{11}\selectfont{i}}}\textcolor[HTML]{000000}{\textsuperscript{\fontsize{11}{11}\selectfont{m}}}} & \multicolumn{1}{!{\color[HTML]{666666}\vrule width 1pt}>{\raggedleft}m{\dimexpr 0.58in+0\tabcolsep}}{\textcolor[HTML]{000000}{\fontsize{11}{11}\selectfont{a}}\textcolor[HTML]{000000}{\textsubscript{\fontsize{11}{11}\selectfont{i}}}} & \multicolumn{1}{>{\raggedleft}m{\dimexpr 1.05in+0\tabcolsep}}{\textcolor[HTML]{000000}{\fontsize{11}{11}\selectfont{V}}\textcolor[HTML]{000000}{\textsubscript{\fontsize{11}{11}\selectfont{i}}}} \\

\ascline{1.5pt}{666666}{1-7}\endfirsthead \caption[]{Accessibility\ values\ at\ each\ origin\ per\ mode\ m\ at\ each\ origin\ i\ and\ aggregated\ between\ modes\ for\ each\ i\ for\ the\ synthetic\ example.}\\

\ascline{1.5pt}{666666}{1-7}

\multicolumn{1}{>{\raggedright}m{\dimexpr 0.88in+0\tabcolsep}}{\textcolor[HTML]{000000}{\fontsize{11}{11}\selectfont{i}}} & \multicolumn{1}{>{\raggedright}m{\dimexpr 0.41in+0\tabcolsep}}{\textcolor[HTML]{000000}{\fontsize{11}{11}\selectfont{m}}} & \multicolumn{1}{!{\color[HTML]{666666}\vrule width 1pt}>{\raggedleft}m{\dimexpr 1.05in+0\tabcolsep}}{\textcolor[HTML]{000000}{\fontsize{11}{11}\selectfont{S}}\textcolor[HTML]{000000}{\textsubscript{\fontsize{11}{11}\selectfont{i}}}\textcolor[HTML]{000000}{\textsuperscript{\fontsize{11}{11}\selectfont{m}}}} & \multicolumn{1}{>{\raggedleft}m{\dimexpr 0.58in+0\tabcolsep}}{\textcolor[HTML]{000000}{\fontsize{11}{11}\selectfont{a}}\textcolor[HTML]{000000}{\textsubscript{\fontsize{11}{11}\selectfont{i}}}\textcolor[HTML]{000000}{\textsuperscript{\fontsize{11}{11}\selectfont{m}}}} & \multicolumn{1}{>{\raggedleft}m{\dimexpr 1.05in+0\tabcolsep}}{\textcolor[HTML]{000000}{\fontsize{11}{11}\selectfont{V}}\textcolor[HTML]{000000}{\textsubscript{\fontsize{11}{11}\selectfont{i}}}\textcolor[HTML]{000000}{\textsuperscript{\fontsize{11}{11}\selectfont{m}}}} & \multicolumn{1}{!{\color[HTML]{666666}\vrule width 1pt}>{\raggedleft}m{\dimexpr 0.58in+0\tabcolsep}}{\textcolor[HTML]{000000}{\fontsize{11}{11}\selectfont{a}}\textcolor[HTML]{000000}{\textsubscript{\fontsize{11}{11}\selectfont{i}}}} & \multicolumn{1}{>{\raggedleft}m{\dimexpr 1.05in+0\tabcolsep}}{\textcolor[HTML]{000000}{\fontsize{11}{11}\selectfont{V}}\textcolor[HTML]{000000}{\textsubscript{\fontsize{11}{11}\selectfont{i}}}} \\

\ascline{1.5pt}{666666}{1-7}\endhead



\multicolumn{1}{>{\raggedright}m{\dimexpr 0.88in+0\tabcolsep}}{} & \multicolumn{1}{>{\raggedright}m{\dimexpr 0.41in+0\tabcolsep}}{\textcolor[HTML]{000000}{\fontsize{11}{11}\selectfont{x}}} & \multicolumn{1}{!{\color[HTML]{666666}\vrule width 1pt}>{\raggedleft}m{\dimexpr 1.05in+0\tabcolsep}}{\textcolor[HTML]{000000}{\fontsize{11}{11}\selectfont{27,292.18}}} & \multicolumn{1}{>{\raggedleft}m{\dimexpr 0.58in+0\tabcolsep}}{\textcolor[HTML]{000000}{\fontsize{11}{11}\selectfont{0.95}}} & \multicolumn{1}{>{\raggedleft}m{\dimexpr 1.05in+0\tabcolsep}}{\textcolor[HTML]{000000}{\fontsize{11}{11}\selectfont{15,696.89}}} & \multicolumn{1}{!{\color[HTML]{666666}\vrule width 1pt}>{\raggedleft}m{\dimexpr 0.58in+0\tabcolsep}}{} & \multicolumn{1}{>{\raggedleft}m{\dimexpr 1.05in+0\tabcolsep}}{} \\





\multicolumn{1}{>{\raggedright}m{\dimexpr 0.88in+0\tabcolsep}}{\multirow[c]{-2}{*}{\parbox{0.88in}{\raggedright \textcolor[HTML]{000000}{\fontsize{11}{11}\selectfont{A}}}}} & \multicolumn{1}{>{\raggedright}m{\dimexpr 0.41in+0\tabcolsep}}{\textcolor[HTML]{000000}{\fontsize{11}{11}\selectfont{z}}} & \multicolumn{1}{!{\color[HTML]{666666}\vrule width 1pt}>{\raggedleft}m{\dimexpr 1.05in+0\tabcolsep}}{\textcolor[HTML]{000000}{\fontsize{11}{11}\selectfont{44,999.80}}} & \multicolumn{1}{>{\raggedleft}m{\dimexpr 0.58in+0\tabcolsep}}{\textcolor[HTML]{000000}{\fontsize{11}{11}\selectfont{1.57}}} & \multicolumn{1}{>{\raggedleft}m{\dimexpr 1.05in+0\tabcolsep}}{\textcolor[HTML]{000000}{\fontsize{11}{11}\selectfont{51,785.72}}} & \multicolumn{1}{!{\color[HTML]{666666}\vrule width 1pt}>{\raggedleft}m{\dimexpr 0.58in+0\tabcolsep}}{\multirow[c]{-2}{*}{\parbox{0.58in}{\raggedleft \textcolor[HTML]{000000}{\fontsize{11}{11}\selectfont{1.36}}}}} & \multicolumn{1}{>{\raggedleft}m{\dimexpr 1.05in+0\tabcolsep}}{\multirow[c]{-2}{*}{\parbox{1.05in}{\raggedleft \textcolor[HTML]{000000}{\fontsize{11}{11}\selectfont{67,482.61}}}}} \\

\ascline{1pt}{666666}{1-7}



\multicolumn{1}{>{\raggedright}m{\dimexpr 0.88in+0\tabcolsep}}{} & \multicolumn{1}{>{\raggedright}m{\dimexpr 0.41in+0\tabcolsep}}{\textcolor[HTML]{000000}{\fontsize{11}{11}\selectfont{x}}} & \multicolumn{1}{!{\color[HTML]{666666}\vrule width 1pt}>{\raggedleft}m{\dimexpr 1.05in+0\tabcolsep}}{\textcolor[HTML]{000000}{\fontsize{11}{11}\selectfont{27,292.18}}} & \multicolumn{1}{>{\raggedleft}m{\dimexpr 0.58in+0\tabcolsep}}{\textcolor[HTML]{000000}{\fontsize{11}{11}\selectfont{0.64}}} & \multicolumn{1}{>{\raggedleft}m{\dimexpr 1.05in+0\tabcolsep}}{\textcolor[HTML]{000000}{\fontsize{11}{11}\selectfont{38,170.03}}} & \multicolumn{1}{!{\color[HTML]{666666}\vrule width 1pt}>{\raggedleft}m{\dimexpr 0.58in+0\tabcolsep}}{} & \multicolumn{1}{>{\raggedleft}m{\dimexpr 1.05in+0\tabcolsep}}{} \\





\multicolumn{1}{>{\raggedright}m{\dimexpr 0.88in+0\tabcolsep}}{\multirow[c]{-2}{*}{\parbox{0.88in}{\raggedright \textcolor[HTML]{000000}{\fontsize{11}{11}\selectfont{B}}}}} & \multicolumn{1}{>{\raggedright}m{\dimexpr 0.41in+0\tabcolsep}}{\textcolor[HTML]{000000}{\fontsize{11}{11}\selectfont{z}}} & \multicolumn{1}{!{\color[HTML]{666666}\vrule width 1pt}>{\raggedleft}m{\dimexpr 1.05in+0\tabcolsep}}{\textcolor[HTML]{000000}{\fontsize{11}{11}\selectfont{44,999.80}}} & \multicolumn{1}{>{\raggedleft}m{\dimexpr 0.58in+0\tabcolsep}}{\textcolor[HTML]{000000}{\fontsize{11}{11}\selectfont{1.05}}} & \multicolumn{1}{>{\raggedleft}m{\dimexpr 1.05in+0\tabcolsep}}{\textcolor[HTML]{000000}{\fontsize{11}{11}\selectfont{94,468.91}}} & \multicolumn{1}{!{\color[HTML]{666666}\vrule width 1pt}>{\raggedleft}m{\dimexpr 0.58in+0\tabcolsep}}{\multirow[c]{-2}{*}{\parbox{0.58in}{\raggedleft \textcolor[HTML]{000000}{\fontsize{11}{11}\selectfont{0.88}}}}} & \multicolumn{1}{>{\raggedleft}m{\dimexpr 1.05in+0\tabcolsep}}{\multirow[c]{-2}{*}{\parbox{1.05in}{\raggedleft \textcolor[HTML]{000000}{\fontsize{11}{11}\selectfont{132,638.94}}}}} \\

\ascline{1pt}{666666}{1-7}



\multicolumn{1}{>{\raggedright}m{\dimexpr 0.88in+0\tabcolsep}}{} & \multicolumn{1}{>{\raggedright}m{\dimexpr 0.41in+0\tabcolsep}}{\textcolor[HTML]{000000}{\fontsize{11}{11}\selectfont{x}}} & \multicolumn{1}{!{\color[HTML]{666666}\vrule width 1pt}>{\raggedleft}m{\dimexpr 1.05in+0\tabcolsep}}{\textcolor[HTML]{000000}{\fontsize{11}{11}\selectfont{2,240.38}}} & \multicolumn{1}{>{\raggedleft}m{\dimexpr 0.58in+0\tabcolsep}}{\textcolor[HTML]{000000}{\fontsize{11}{11}\selectfont{0.68}}} & \multicolumn{1}{>{\raggedleft}m{\dimexpr 1.05in+0\tabcolsep}}{\textcolor[HTML]{000000}{\fontsize{11}{11}\selectfont{2,035.86}}} & \multicolumn{1}{!{\color[HTML]{666666}\vrule width 1pt}>{\raggedleft}m{\dimexpr 0.58in+0\tabcolsep}}{} & \multicolumn{1}{>{\raggedleft}m{\dimexpr 1.05in+0\tabcolsep}}{} \\





\multicolumn{1}{>{\raggedright}m{\dimexpr 0.88in+0\tabcolsep}}{\multirow[c]{-2}{*}{\parbox{0.88in}{\raggedright \textcolor[HTML]{000000}{\fontsize{11}{11}\selectfont{C}}}}} & \multicolumn{1}{>{\raggedright}m{\dimexpr 0.41in+0\tabcolsep}}{\textcolor[HTML]{000000}{\fontsize{11}{11}\selectfont{z}}} & \multicolumn{1}{!{\color[HTML]{666666}\vrule width 1pt}>{\raggedleft}m{\dimexpr 1.05in+0\tabcolsep}}{\textcolor[HTML]{000000}{\fontsize{11}{11}\selectfont{3,745.89}}} & \multicolumn{1}{>{\raggedleft}m{\dimexpr 0.58in+0\tabcolsep}}{\textcolor[HTML]{000000}{\fontsize{11}{11}\selectfont{1.12}}} & \multicolumn{1}{>{\raggedleft}m{\dimexpr 1.05in+0\tabcolsep}}{\textcolor[HTML]{000000}{\fontsize{11}{11}\selectfont{7,842.59}}} & \multicolumn{1}{!{\color[HTML]{666666}\vrule width 1pt}>{\raggedleft}m{\dimexpr 0.58in+0\tabcolsep}}{\multirow[c]{-2}{*}{\parbox{0.58in}{\raggedleft \textcolor[HTML]{000000}{\fontsize{11}{11}\selectfont{0.99}}}}} & \multicolumn{1}{>{\raggedleft}m{\dimexpr 1.05in+0\tabcolsep}}{\multirow[c]{-2}{*}{\parbox{1.05in}{\raggedleft \textcolor[HTML]{000000}{\fontsize{11}{11}\selectfont{9,878.45}}}}} \\

\ascline{1pt}{666666}{1-7}



\multicolumn{1}{>{\raggedright}m{\dimexpr 0.88in+0\tabcolsep}}{\textcolor[HTML]{000000}{\fontsize{11}{11}\selectfont{TOTALS}}} & \multicolumn{1}{>{\raggedright}m{\dimexpr 0.41in+0\tabcolsep}}{\textcolor[HTML]{000000}{\fontsize{11}{11}\selectfont{}}} & \multicolumn{1}{!{\color[HTML]{666666}\vrule width 1pt}>{\raggedleft}m{\dimexpr 1.05in+0\tabcolsep}}{\textcolor[HTML]{000000}{\fontsize{11}{11}\selectfont{150,570.22}}} & \multicolumn{1}{>{\raggedleft}m{\dimexpr 0.58in+0\tabcolsep}}{\textcolor[HTML]{000000}{\fontsize{11}{11}\selectfont{N/A}}} & \multicolumn{1}{>{\raggedleft}m{\dimexpr 1.05in+0\tabcolsep}}{\textcolor[HTML]{000000}{\fontsize{11}{11}\selectfont{210,000.00}}} & \multicolumn{1}{!{\color[HTML]{666666}\vrule width 1pt}>{\raggedleft}m{\dimexpr 0.58in+0\tabcolsep}}{\textcolor[HTML]{000000}{\fontsize{11}{11}\selectfont{N/A}}} & \multicolumn{1}{>{\raggedleft}m{\dimexpr 1.05in+0\tabcolsep}}{\textcolor[HTML]{000000}{\fontsize{11}{11}\selectfont{210,000.00}}} \\

\ascline{1.5pt}{666666}{1-7}



\end{longtable}



\arrayrulecolor[HTML]{000000}

\global\setlength{\arrayrulewidth}{\Oldarrayrulewidth}

\global\setlength{\tabcolsep}{\Oldtabcolsep}

\renewcommand*{\arraystretch}{1}

The calculated \(S_i^m\), \(a_i^m\) and \(V_i^m\) accessibility values
for each \(i\) and \(m\) are shown in the middle three columns and are
aggregated for each \(i\) in the final two columns in Table 1 . We use a
negative exponential impedance function
\(f(c_{ij}) = \exp(-\beta\cdot c_{ij})\) with \(\beta=0.1\) for both
\(x\) and \(z\) modes for all accessibility measures calculations.

The Hansen-type measure \(S_i^m\) is presented for each origin and mode
in third column of Table 1 . For all \(i\), the \(z\)-using
sub-population has higher \(S_i^m\) values than the \(x\)-using
sub-populations. Additionally, \(S_i^m\) is equal for both mode-using
populations in \(A\) and \(B\). This is the case because \(S_i^m\) does
not consider \emph{competition}, it only relies on reflecting the count
of opportunities that may be interacted with as a product of
\(f^m(c_{ij}^m)\). Recall, populations in \(A\) and \(B\) have the same
travel impedance to employment centers \(1\), \(2\) and \(3\) (either
15, 30, or 100 minutes using \(x\) or 10, 25, or 80 minutes using
\(z\)). As such, these the calculated \(S_i^m\) values are the same for
both \(A\) and \(B\). Furthermore, the total sum of \(S_i^m\) in the
region is equal to 150,570.2. This value is difficult to interpret: it
represents the weighted sum of opportunities that may be interacted with
within the region based on travel impedance. It cannot be interpreted as
any sort of benchmark since the measure is \emph{non-constrained}. To
connect this example to literature, \(S_i^m\) is calculated in the work
of {[}15{]}; they compare differences in \(S_i^m\) values between modes
in a relative and comparative sense, but make no further interpretation
of the \(S_i^m\) values.

In the fourth and sixth column in Table 1 the Shen-type measure is
calculated: first for both origin and mode \(a_i^m\) as well as
aggregated by the weighted mean mode-population (
\(\sum_m \frac{P_i^m}{P_i}*a_i^m\) ) to represent a value for each
origin \(a_i\). Unlike \(S_i^m\), this measure considers
\emph{competition}. For instance, the \(x\)-using populations in \(A\)
and \(B\) centers do not have the same \(a_i^m\) values as the
\(z\)-using. In fact, \(A\) has the highest values \(a_i^m\) and \(a_i\)
values since this center has the smallest travel impedance to
opportunities (lower than at \(C\), \(A\) and \(B\) are equal) and has
one of these lowest proximity to a relatively high amount of population
(lower than at \(B\)).

However, the Shen-type measure is \emph{non-constrained}: the total sum
of \(a_i^m\) or \(a_i\) is practically meaningless since it represents a
sum of ratios. For instance, the \(z\)-using sub-population at \(A\) has
a value of 1.57 potential jobs per potential job-seeking population
compared to 0.95 for \(x\)-using sub-population. What is the
significance of these values? The difference between these modes is
equal to 0.62, but 0.62 of what? How many more job opportunities are
\(z\) users interacting with than \(x\) users? When \(a_i^m\) is
aggregated to \(a_i\) as shown in the sixth column, the values face
similar interpretability issues. The Shen-type measure is implemented in
the previously discussed work of {[}20{]} to calculate modal \(a_i^m\)
values and the aggregated \(a_i\) is implemented in the work of
{[}21{]}. However, similar to the Hansen-type measure, these works
discuss relative and spatially comparative differences in values, they
do not make further interpretation of the \(a_i^m\) or \(a_i\)
themselves. This may be because the Shen-type measure is
\emph{non-constrained}, this is no benchmark or global maximum to which
comparisons can be drawn from.

By contrast, spatial availability \(V_i\) considers competition and is
constrained such that the total sum of values is equal to the total
number of opportunities in the region (i.e., \(210,000\) jobs). Seen in
fifth column of Table 1 , \(V_i^m\) for the same mode-using populations
in \(A\) and \(B\) are not the same (as this measure considers
competition). In fact, at \(A\), the \(z\)-using sub-population captures
36,088.84 more spatially available jobs (of the \(210,000\) jobs in the
region) than the sub-population using mode \(x\). The numerical
difference has a practical interpretation.

Furthermore, \(V_i^m\) values for an \(i\) can be aggregated across
\(m\) and compared across \(i\) ( \(V_i = \sum_m{\sum_i{V_i^m}}\) ) as a
result of the proportional allocation mechanism. This aggregation,
\(V_i\), is shown in the seventh column in Table 1 . Again looking at
center \(A\), \(A\) is allocated 67,482.61 spatially available
opportunities for both modes. 77\% of this spatial availability
allocated to \(A\) is assigned to the \(z\)-using population despite
representing 66\% of \(A\)'s population.

Spatial availability can be further aggregated to better interpret
competition between modes. Across the entire region, 130,000 people use
\(z\) (62\% of the region population). However, the \(z\)-using
population accounts for 73\% of the region's total spatial availability
- the rest is allocated to the \(x\)-using population (38\%of the total
population). Notably, the \(x\)-using population captures 11\% less
spatial availability to opportunities than its population proportion.
This understanding can lead us to ask normative questions such as, how
unequal should opportunity access for the two mode-using populations be?
Can the lower-travel-cost populations spare some spatial availability if
a policy of modal-restriction (like a LEZ) was introduced?

Since spatial availability is constrained and has an interpretable
meaning as a proportion of the total opportunities in the region, the
values at \(i\) have a new significance. Inequality in \(V_i^m\) values
can be explored through a variety of approaches. For instance, consider
travel times. The \(z\)-using population accounts for 67\% of the
potential travel time traveled in the region: this is 7\% less travel
time than the proportion of spatial available opportunities that is
allocated to them. In other words, the \(z\)-using population travels
less minutes overall and has more spatial availability of opportunities
than the \(x\)-using population using the slower mode \(x\).

Alternatively, inequities in spatial availability between mode-using
populations can be explored through proportional benchmarks. A spatial
availability per capita \(v_i^m\) as presented in Equation
(\ref{eq:SA-per-capita}):

\begin{equation}
\label{eq:SA-per-capita}
v_{i}^m = \frac{V_{i}^m}{P_{i}^m}
\end{equation}

The \(v_i^m\) values for \(A\), \(B\), and \(C\) for the \(x\)-using
sub-populations are 0.95, 0.64 and 0.68 spatially available jobs per
capita, respectively. The \(v_i^m\) for the \(z\)-using sub-populations
are much higher, with values of: 1.57, 1.05 and 1.12 respectively. The
\(x\)-using population, especially at \(B\) and \(C\), are directly
impacted by the jobs that are spatially available to the \(z\)-using
population \emph{in addition to} the mass effect (occurring at \(B\),
high population density) and high travel impedance (occurring at the
Satellite \(C\)).

If, lets say, the planning goal is to have one spatially available job
per mode-using population, a policy intervention can be put in place, to
reduce the \(v_i^z\) values and increase \(v_i^x\) values. This
demonstration is to show how simply the \(V_i^m\) framework can be
manipulated quantify the competitive (dis)advantage in a multimodal
application. In what follows, we further explore competition between
multiple modes through an empirical example.

\hypertarget{empirical-example-madrid-lez}{%
\section{Empirical example: Madrid
LEZ}\label{empirical-example-madrid-lez}}

\hypertarget{multimodal-data-and-methods}{%
\subsection{Multimodal data and
methods}\label{multimodal-data-and-methods}}

Low emission zones (LEZ) have been implemented as a climate change
policy intervention to reduce GHG emissions, improve air quality, and
support sustainable mobility in many countries. Though rules vary, LEZ
aim to deter/reduce traffic in designated zones under threat of penalty
(e.g., fines, seizure of vehicle). From the perspective of restriction
for passenger transport, LEZ are a policy of \emph{geographic
discrimination} as they change how people access opportunities by making
the travel impedance more costly for car-mode users. If seeing
opportunities as finite, this discrimination allows populations to
access opportunities by other modes more readily than before. In this
way, LEZ change the multimodal competitive accessibility landscape of a
city.

Spain is one of a few countries with active LEZ and plans to expand
their implementation as specified in their climate-change-related plans:
\emph{Plan Nacional Integrado de Energía y Clima 2021-2030} {[}22{]} and
\emph{Plan Nacional de Control de la Contaminación Atmosférica}
{[}23{]}. Specifically, the national Spanish law 7/2021 ( \emph{Ley de
Cambio Climático y Transición Energética}) will require all
municipalities to implement LEZ by 2023 if they meet at least one of the
following requirements: (i) municipalities \textgreater50,000 inhab.;
(ii) islands; and (iii) municipalities \textgreater{} 20,000 inhab. when
air quality exceeds limits specified in \emph{RD 102/2011 de Mejora de
Calidad del Aire} {[}24{]}.

In 2017, LEZs were implemented in the Spanish capital city of Madrid
following the goals set out in the national agenda . In geographic
scope, the 2017 boundaries of the LEZ were relatively small (covering
4.72 km\^{}\{2\}) and within the center (i.e., LEZ Centro). These
boundaries were expanded in 2023 to inside of the M-30, a highway in
proximity to the city center (i.e., LEZ M-30) and the city has plans to
further spatially expand the LEZ. Within the 2017 LEZ Centro
implementation, all cars, motorcycles and freight with environmental
label A or B (higher polluting classification, associated with older
make and model of fossil fuel internal combustion engine vehicles), are
not permitted to enter the area unless they are used by residents or
meet other exemptions. This restriction impacted approximately half of
all car trips that were typically made into the LEZ Centro {[}25{]}.

For this case study, we use \(V_i^m\) to quantify the competition of
spatially available opportunities between modes after the LEZ Centro
implementation. Particularly, we demonstrate how \(V_i^m\) can be used
to spectate on how the restriction of car mobility in areas
around/within the LEZ Centro allowed the other, more sustainable but
often with higher travel impedance modes, to become more competitive.

\begin{figure}

{\centering \includegraphics[width=1\linewidth]{images/i_jobs_zn208_plot} 

}

\caption{\label{fig:Fig2} Jobs $O_j$ taken by people living and working in Madrid as reported by the 2018 travel survey.}\label{fig:jobs-plot}
\end{figure}

\begin{figure}

{\centering \includegraphics[width=1\linewidth]{images/im_populations_zn208_plot} 

}

\caption{\label{fig:Fig3} Population living and working in Madrid, by four summarized modal categories, $P^m_i$ as reported by the 2018 travel survey.}\label{fig:pop-plot}
\end{figure}

The 2018 Community of Madrid travel survey ({[}26{]}) is the source of
data for this empirical example: it is a representative survey that
reflects a snap-shot of the travel patterns for one typical day of the
working week (e.g., n=222,744 trips with representative population
elevation factors). In this paper, a sample of the travel survey is
used, namely the residential home origin to work destination trips of
all modes and those that originate and end in the city of Madrid. These
totals are displayed in Figure \ref{fig:Fig2} and Figure \ref{fig:Fig3}.
Both figures are displayed at the level of traffic analysis zones (\(i\)
and \(j\)) that correspond to the survey. The red boundary represents
the LEZ Centro in effect in 2017 and thus those travel patterns of
car-restriction reflected in the survey. The cyan boundary represents
the LEZ that will be within the boundaries of the M-30 highway in 2023
and is present in the plots as a spatial reference for areas in
proximity to the LEZ Centro.

The total sum of jobs \(O_j\) that are held are shown in Figure
\ref{fig:Fig2} and the populations that go to a work destination by four
modal categories \(P^m_i\), is reflected in Figure \ref{fig:Fig3}. The
modal categories represented in Figure \ref{fig:Fig3} are summarized for
the following trip mode types:

\begin{itemize}
\tightlist
\item
  Car/motor: all cars and operating modes (e.g., cab, private driver,
  company, rental care, main driver, passenger, etc.) and all public,
  private or company motorcycle/mopeds.
\item
  Transit: all bus, trams, and trains,
\item
  Bike: all bicycle trips (e.g., private, public, or company bike trips)
  and ``other'' types of micromobility options,
\item
  Walk: walking or by foot,
\end{itemize}

From Figure \ref{fig:Fig2}, it can be seen that the largest
concentration of jobs are within, near, and to the north of the LEZ
Centro. The population that is accessing those jobs by mode (Figure
\ref{fig:Fig3}), appear spatially distinct. Car and transit trips
represent 37\% and 47\% of the modal share respectively. The population
that travels using transit is more spatially distributed than those
using cars - particularly near and within LEZ Centro. This distribution
could be a result of a variety of factors including: transit coverage
and service within with city, effective car infrastructure outside of
the M-30, and/or the impact of the Central LEZ itself.

From Figure \ref{fig:Fig3}, it can also be seen that biking and walking
trips are less common than motorized trips at 1\% and 15\% respectively.
Noteably, there is a positive trend between the populations of walking
and biking trips in zones and populations of transit trips. This
positive trend is higher than for car trip populations.

The travel time for each trip is provided within the survey. These
travel times, per modal category, are used to calibrate mode-specific
travel impedance functions \(f^m(c_{ij}^m)\). To illustrate the modal
differences in travel lengths, summary descriptive per mode are
detailed:

\begin{itemize}
\tightlist
\item
  Car/motor: 36 min (min:0 min., Q2: 15 min., Q3: 55 min., max: 120
  min.)
\item
  Transit: 55 min. (min:1 min., Q2: 30 min., Q3: 80 min., max: 120 min.)
\item
  Bike: 34 min. (min:5 min., Q2: 15 min., Q3: 40 min., max: 115 min.)
\item
  Walk: 27 min. (min:1 min., Q2: 10 min., Q3: 45 min., max: 119 min.)
\end{itemize}

To calculate \(f^m(c_{ij}^m)\) from the survey travel times, a concept
known as the trip length distribution (TLD) was used. A TLD represents
the proportion of trips that are taken at a specific travel cost such as
travel time (i.e., probability density distribution of trips taken by
travel cost). This distribution is then used to derive impedance
functions (e.g., done in the accessibility works of {[}27{]}, {[}28{]},
and {[}29{]}). Maximum likelihood estimation and the Nelder-Mead method
for direct optimization available within the R \{fitdistrplus\} package
{[}30{]} is used to fit the impedance functions. As shown as shown in
Figure \ref{fig:Fig4}, based on goodness-of-fit criteria and associated
diagnostics, the gamma and log-normal probability density function (line
curves) are selected as best fitting curves for the motorized and
non-motorized modes respectively. The selection of functional form
aligns with empirical examples in other regions {[}14,31,32{]}. Overall,
the plots in Figure \ref{fig:Fig4} display the probability of travel
given a trip travel time, based on trip flows from the survey. These
`probability of travel' at each travel time for each mode are realized
observations that reflect the land-use, the transport system, and the
population travel behaviour in Madrid.

\begin{figure}

{\centering \includegraphics[width=1\linewidth]{images/tlds_curves_m_plot} 

}

\caption{\label{fig:Fig4} Fitted impedance function curve (line) against empirical TLD (bars) corresponding to the home-to-work origin-destination flows from the Madrid 2018 travel survey.}\label{fig:tlds-curves-m-plot}
\end{figure}

\hypertarget{results}{%
\subsection{Results}\label{results}}

The spatial availability of jobs \(V_i^m\) is calculated for each of the
four modal categories \(m\) at the level of traffic analysis zones \(i\)
in Madrid and demonstrated in Figure \ref{fig:Fig5}.

\begin{figure}

{\centering \includegraphics[width=1\linewidth]{images/SA_im_V_zn208_plot} 

}

\caption{\label{fig:Fig5} Spatial availability of job opportunities per origin and mode $V_i^m$ in Madrid. Calculated using the home-to-work origin-destination flows from the 2018 travel survey. }\label{fig:SA-m-plot}
\end{figure}

\(V_i^m\) is a proportion of the total number of the 847,574 jobs in the
region and is visualized in Figure \ref{fig:Fig5}. Since \(V_i^m\) is
calculated based on the likelihood of travel from observed home-to-work
journeys, the values can be understood as the number of full-time jobs
that are spatially available to the full-time working population at that
\(i\) and their associated \(m\), relative to all the jobs in the city.
\(V_i^m\) is the number of jobs that are \emph{spatially available} to a
\(m\)-using population located at \(i\), relative to the travel
impedance and size of \emph{all} populations in the region.

Notable are the differences in the magnitude of \(V_i^m\) between modes
in Figure \ref{fig:Fig5}. The majority of \(V_i^m\) is allocated to car-
and transit- using populations. This is to be expected, as the
population that commutes using these modes represents 84.1\% of the
total population. Differences in \(V_i^m\) values within mode-using
populations also exist: car-using populations outside of the M-30 region
appear to have greater \(V_i^m\) values, while some \(i\) areas inside
the M-30 appear to have higher \(V_i^m\) values for the transit-using
populations. Overall, the magnitude of \(V_i^m\) values for the bikers
and walkers are lower than car and transit but the highest
\(V_i^{bike}\) and \(V_i^{walk}\) values tend to be allotted to \(i\)s
within the M-30 and \(i\)s that have higher \(V_i^{transit}\) values.

The differences between the mode-using population and their
mode-specific spatial availability highlights the competitive advantage
offered to certain modes in certain spatial extents. As summarized in
the left-most columns in Figure \ref{fig:Fig6}, the `car/motor' and
`transit' populations represent a combined 95.3\% of the total spatial
availability in the city. However, the `car/motor' using population is
allocated disproportionately more \(V_i^m\) than its size compared to
the transit-using population. The car-using and transit-using population
is 36.6\% and 47.5\% respectively, but is allocated 48.0\% and 47.3\%
respectively, of the city's spatial availability. When treating the
number of opportunities that can be reached as a finite value (total:
847,574 opportunities), fewer opportunities are spatial availability to
the lesser competitive modes-using populations, in this case walking and
cycling. These modes are less competitive as a result of: their lower
travel impedance values at longer travel times (see Figure
\ref{fig:Fig4} at travel times beyond \textasciitilde30 minutes); their
low population values values overall; and higher populations present in
origins with high motorized mode commuting. These factors all contribute
to the the car/motor mode being most advantaged in capturing spatially
available job opportunities overall.

\begin{figure}

{\centering \includegraphics[width=1\linewidth]{images/modal_V_comps_4plot} 

}

\caption{\label{fig:Fig6} Displays the proportion of the working population by mode and spatial availability of job opportunities by mode aggregated for three spatial areas. From left to right, the city of Madrid (All), the area within the M-30 highway (M-30),the area within the Centro region (Centro).}\label{fig:modal-V-comps-plot}
\end{figure}

There are spatial variations in the competitive advantage of the
car-using populations. The proportion of car-using population in the
Centro is smaller and has higher travel impedance values relative to the
inputs in other areas and mode-using populations. The LEZ Centro
implementation further restricts the car-advantage as it shifted more
than half of all car trips into the LEZ to another mode {[}25{]}. This
restriction decreased the number of car-using population from \(i\)s
going into the LEZ Centro (an area with a large number of jobs overall,
see Figure \ref{fig:Fig2}), thus increasing the mass effect for non-car
modes and resulting in proportionally higher \(v_i^m\) values for
non-car modes. As such, the lower amount of access to opportunities by
car-mode allows more opportunities in the LEZ to be available by
populations using other modes.

As summarized in the two right-most columns in Figure \ref{fig:Fig6},
the proportion of spatial availability allocated to the car-using
population in the Centro (13.8\% or 5,373 opportunities). As a
comparative reference, this is less than the proportion of the car-using
population in the Centro (16.1\%), evidently less then the proportion of
car-using population in the city, and is the opposite of the trend
overall (left-most columns) and within the M-30 (middle columns). More
opportunities are spatial availability to non-car using populations
within the Centro, particularly transit-using populations (68.5\% of
spatially available jobs in the Centro despite representing 51.4\% of
the population in the Centro and 47.5\% in the city overall).

From Figure \ref{fig:Fig6}, it is also summarized that there is a higher
proportion of opportunities spatially available to walking and cycling
populations in the Centro than in the City overall and in all areas
within the M-30. Notably, within the Centro, 1.7\% and 16.1\% of
opportunities are spatially available to bike and walk modes
respectively, while their populations represent smaller proportions of
1.2\% and 14.7\% of the population overall. Though the proportion of
spatial availability for these mode-using populations is still lower
than the proportion of mode-using population located in the Centro,
these modes are more competitive within the Centro than outside of the
Centro. By restricting the more competitive car mode through the LEZ,
the advantage in the spatial availability of opportunities afforded to
the otherwise lesser competitive modes is made apparent.

The spatial differences in the competitive dis/advantage of spatial
availability between modes can also be visualized per origin. Figure
\ref{fig:Fig7} visualizes \(v_i^m\), the spatial availability \(V_i^m\)
divided by the mode-population. \(v_i^m\) values above 1 are represented
in increasing red shades, values below 1 are represented in increasingly
green shades, and values equal to 1 are white. These plots illustrates
the discussion of the disproportionately high over-allocation of spatial
availability relative to the mode-using population in many of the
origins for the car/motor mode (bottom left plot, areas denoted with
green \(v_i^m\) values above 1). These plots also visualize areas that
disproportionately capture lower spatial availability (under 1),
represented in shades of red. It can be observed that the transit-using
population's spatial availability to jobs is relatively balanced (i.e.,
many zones are white), while the non-motorized modes \(v_i^m\) values
are low (under 1) overall.

\begin{figure}

{\centering \includegraphics[width=1\linewidth]{images/SA_im_vv_zn208_plot} 

}

\caption{\label{fig:Fig7} Spatial availability of job opportunities per mode-using capita by mode $v_i^m$ per origin in Madrid. Calculated using the home-to-work origin destination flows from the 2018 travel survey.}\label{fig:SA-per-capita-m-plot}
\end{figure}

Interestingly, as also represented in Figure \ref{fig:Fig6}, \(v_i^m\)
for car/motor within and near the LEZ Centro is near or below 1
(white/red) in Figure \ref{fig:Fig7} while all non-car modes have
relatively higher \(v_i^m\) values. Though the spatial availability from
before the LEZ Centro implementation is unknown, Figure \ref{fig:Fig7}
provides a benchmark for quantifying potential LEZ implementations in
the future (given 2018 travel conditions). Figure \ref{fig:Fig7} also
shows that many areas within the M-30 have high (white/green) \(v_i^m\)
values for car-mode, signaling that the spatial expansion of the LEZ
Centro stands to increase the spatial availability of jobs for non-car
mode using populations.

\hypertarget{discussion-and-conclusions}{%
\section{Discussion and conclusions}\label{discussion-and-conclusions}}

Location-based accessibility measures like the Hansen-type \(S_i^m\),
Shen-type \(a_i^m\), and spatial availability \(V_i^m\) measures share a
commonality; they are a weighted sum of opportunities assigned to each
spatial unit \(i\) in a region. In this way, they all can be interpreted
as a score of how many opportunities can be potentially interacted with
by the population at \(i\). How the weight and sum of the
potentially-interacted-with opportunities is considered is what defines
the type of accessibility measure.

Within this paper, the location-based singly- \emph{constrained} and
\emph{competitive} accessibility measure, known as spatial availability
\(V_i\) {[}14{]}, is extended for the case of capturing multimodal
accessibility to opportunities \(V_i^m\). A synthetic example and then
an empirical case of LEZ in Madrid are detailed to demonstrate this
multimodal extension.

The spatial availability measure is capable of capturing a new
interpretation of multimodal competition that previous accessibility
measures have not yet done. Competitive measures hypothesis that
populations using modes with lower travel impedance, when competing for
a finite set of opportunities, will capture more opportunities. With
spatial availability, the number of opportunities that are captured (of
the total opportunities in the region) by each mode can be individually
calculated. From there, the difference between how many spatially
available opportunities one mode captures versus another can be
investigated. This is the advantage of the spatial availability measure,
particularly its multimodal extension.

The flexibility and need for an accessibility measure such as spatial
availability is pertinent in policy scenario evaluation. As showcased in
the empirical example of the LEZ in Madrid, competition for job
opportunity availability varies spatially \emph{as well as} between
modes. The car and transit modes have the highest spatial availability,
with the car-mode having highest availability with exception to the
areas within the LEZ Centro. Since car travel has been highly restricted
within the LEZ Centro, fewer car-using people potentially interact with
jobs within the LEZ Centro, leaving more \emph{spatially available} jobs
for non-car-using populations. This difference in car-using populations
in locations accessing jobs within and immediately outside the LEZ
Centro increases the competitiveness of the transit-using population
(the second most competitive mode) as well as the non-motorized modes.

Spatial availability \(V_i^m\) can also be divided by the mode-using
population at each \(i\) to yield mode-population normalized values.
These values, reflected in Figure \ref{fig:Fig7}, can be used as a
benchmark to investigate existing conditions and plan future LEZ
implementation (i.e., target areas with exceptionally high car spatial
availability such that more opportunities are available to other
mode-users).

In summary, conventional \emph{non-constrained} accessibility measures
are difficult for planners to operationalize for a variety of reasons
including issues of computation and interpretability {[}2{]}. With
spatial availability, the magnitude of opportunities that are available
as a proportion of all the opportunities in the region is equal to
\(V_i\). As a result of its proportional allocation mechanism, \(V_i\)
can be naturally extended into multimodal applications. This flexibility
is helpful to modelling policy scenarios in our cities that are
increasingly multimodal. The interpretation of \(V_i\) allows for
manipulation of \(V_i^m\) values to investigate differences of
availability between neighbourhoods, modes, and regions, generate per
capita benchmarks, and/or generate average values per population-group.

From a spatial equity perspective, spatial availability measure can
provide researchers, policy makers, and citizens a new-found
interpretation of accessibility measures. With a plot of spatial
availability values, one can begin asking, how much is enough and what
level may be too much. These interpretations were difficult to be made
with accessibility measures in the past.

\hypertarget{acknowledgements}{%
\section{Acknowledgements}\label{acknowledgements}}

This research was funded by the Canada Graduate Scholarship - Doctoral
Program (CGS D) provided by the Social Sciences and Humanities Research
Council (SSHRC) and Project Mobilizing Justice, also supported by SSHRC.
All work is fully-reproducible and available within this
\href{https://github.com/soukhova/Madrid-ZBE-LEZ}{GitHub repository}.

\hypertarget{author-contributions}{%
\section{Author contributions}\label{author-contributions}}

The authors confirm contribution to the paper as follows: study
conception and design: AS, JTO, JSL, AP.; data collection: AS, JTO,
JSL.; analysis and interpretation of results: AS, JTO, JSL, AP.; draft
manuscript preparation: AS, JSL, AP. All authors reviewed the results
and approved the final version of the manuscript.

\hypertarget{references}{%
\section*{References}\label{references}}
\addcontentsline{toc}{section}{References}

\hypertarget{refs}{}
\begin{CSLReferences}{0}{0}
\leavevmode\vadjust pre{\hypertarget{ref-hansenHowAccessibilityShapes1959}{}}%
\CSLLeftMargin{1. }%
\CSLRightInline{Hansen WG. How accessibility shapes land use. Journal of
the American Institute of Planners. 1959;25: 73--76.
doi:\href{https://doi.org/10.1080/01944365908978307}{10.1080/01944365908978307}}

\leavevmode\vadjust pre{\hypertarget{ref-levinsonTransportAccessManual2020}{}}%
\CSLLeftMargin{2. }%
\CSLRightInline{Levinson D, King D. Transport access manual: {A} guide
for measuring connection between people and places. {University of
Sydney}; 2020. Available:
\url{https://ses.library.usyd.edu.au/handle/2123/23733}}

\leavevmode\vadjust pre{\hypertarget{ref-gowerPlanningInnovationCity2022}{}}%
\CSLLeftMargin{3. }%
\CSLRightInline{Gower A, Grodach C. Planning innovation or city
branding? Exploring how cities operationalise the 20-minute
neighbourhood concept. Urban Policy and Research. 2022;40: 36--52.
doi:\href{https://doi.org/10.1080/08111146.2021.2019701}{10.1080/08111146.2021.2019701}}

\leavevmode\vadjust pre{\hypertarget{ref-siddiqToolsTradeAssessing2021}{}}%
\CSLLeftMargin{4. }%
\CSLRightInline{Siddiq F, D. Taylor B. Tools of the trade?: Assessing
the progress of accessibility measures for planning practice. Journal of
the American Planning Association. 2021;87: 497--511.
doi:\href{https://doi.org/10.1080/01944363.2021.1899036}{10.1080/01944363.2021.1899036}}

\leavevmode\vadjust pre{\hypertarget{ref-yanAccessibilityBasedPlanningAddressing2021}{}}%
\CSLLeftMargin{5. }%
\CSLRightInline{Yan X. Toward accessibility-based planning addressing
the myth of travel cost savings. {JOURNAL} {OF} {THE} {AMERICAN}
{PLANNING} {ASSOCIATION}. 2021;87: 409--423.
doi:\href{https://doi.org/10.1080/01944363.2020.1850321}{10.1080/01944363.2020.1850321}}

\leavevmode\vadjust pre{\hypertarget{ref-fikselSustainabilityResilienceSystems2006}{}}%
\CSLLeftMargin{6. }%
\CSLRightInline{Fiksel J. Sustainability and resilience: Toward a
systems approach. Sustainability: Science, Practice and Policy. 2006;2:
14--21.
doi:\href{https://doi.org/10.1080/15487733.2006.11907980}{10.1080/15487733.2006.11907980}}

\leavevmode\vadjust pre{\hypertarget{ref-valencaMainChallengesOpportunities2021}{}}%
\CSLLeftMargin{7. }%
\CSLRightInline{Valença G, Moura F, Morais De Sá A. Main challenges and
opportunities to dynamic road space allocation: From static to dynamic
urban designs. Journal of Urban Mobility. 2021;1: 100008.
doi:\href{https://doi.org/10.1016/j.urbmob.2021.100008}{10.1016/j.urbmob.2021.100008}}

\leavevmode\vadjust pre{\hypertarget{ref-leeMeasuringImpactsNew2018}{}}%
\CSLLeftMargin{8. }%
\CSLRightInline{Lee J, Miller HJ. Measuring the impacts of new public
transit services on space-time accessibility: An analysis of transit
system redesign and new bus rapid transit in columbus, ohio, {USA}.
Applied Geography. 2018;93: 47--63.
doi:\href{https://doi.org/10.1016/j.apgeog.2018.02.012}{10.1016/j.apgeog.2018.02.012}}

\leavevmode\vadjust pre{\hypertarget{ref-mohriClusteringMethodMeasuring2021}{}}%
\CSLLeftMargin{9. }%
\CSLRightInline{Mohri S, Mortazavi S, Nassir N. A clustering method for
measuring accessibility and equity in public transportation service:
{Case} study of {Melbourne}. SUSTAINABLE CITIES AND SOCIETY. 2021;74.
doi:\href{https://doi.org/10.1016/j.scs.2021.103241}{10.1016/j.scs.2021.103241}}

\leavevmode\vadjust pre{\hypertarget{ref-maoMeasuringSpatialAccessibility2013}{}}%
\CSLLeftMargin{10. }%
\CSLRightInline{Mao L, Nekorchuk D. Measuring spatial accessibility to
healthcare for populations with multiple transportation modes. Health \&
Place. 2013;24: 115--122.
doi:\href{https://doi.org/10.1016/j.healthplace.2013.08.008}{10.1016/j.healthplace.2013.08.008}}

\leavevmode\vadjust pre{\hypertarget{ref-tarrinoortizPublicAcceptabilityLow2021}{}}%
\CSLLeftMargin{11. }%
\CSLRightInline{Tarriño-Ortiz J, Soria-Lara JA, Gómez J, Vassallo JM.
Public acceptability of low emission zones: The case of {``madrid
central.''} Sustainability. 2021;13: 3251.
doi:\href{https://doi.org/10.3390/su13063251}{10.3390/su13063251}}

\leavevmode\vadjust pre{\hypertarget{ref-devrijNooneVisitsMe2022}{}}%
\CSLLeftMargin{12. }%
\CSLRightInline{De Vrij E, Vanoutrive T. {``No-one visits me anymore''}:
Low emission zones and social exclusion via sustainable transport
policy. 2022 {[}cited 27 Jul 2023{]}. Available:
\url{https://www.tandfonline.com/doi/epdf/10.1080/1523908X.2021.2022465?needAccess=true\&role=button}}

\leavevmode\vadjust pre{\hypertarget{ref-verbeekJustManagementUrban2022}{}}%
\CSLLeftMargin{13. }%
\CSLRightInline{Verbeek T, Hincks S. The {``just''} management of urban
air pollution? A geospatial analysis of low emission zones in brussels
and london. Applied Geography. 2022;140: 102642.
doi:\href{https://doi.org/10.1016/j.apgeog.2022.102642}{10.1016/j.apgeog.2022.102642}}

\leavevmode\vadjust pre{\hypertarget{ref-soukhovIntroducingSpatialAvailability2023}{}}%
\CSLLeftMargin{14. }%
\CSLRightInline{Soukhov A, Paez A, Higgins CD, Mohamed M. Introducing
spatial availability, a singly-constrained measure of competitive
accessibility {\textbar} {PLOS} {ONE}. {PLOS} {ONE}. 2023; 1--30.
doi:\href{https://\%20doi.org/10.1371/journal.pone.0278468}{https://
doi.org/10.1371/journal.pone.0278468}}

\leavevmode\vadjust pre{\hypertarget{ref-tahmasbiMultimodalAccessibilitybasedEquity2019}{}}%
\CSLLeftMargin{15. }%
\CSLRightInline{Tahmasbi B, Mansourianfar MH, Haghshenas H, Kim I.
Multimodal accessibility-based equity assessment of urban public
facilities distribution. Sustainable Cities and Society. 2019;49:
101633.
doi:\href{https://doi.org/10.1016/j.scs.2019.101633}{10.1016/j.scs.2019.101633}}

\leavevmode\vadjust pre{\hypertarget{ref-luoMeasuresSpatialAccessibility2003}{}}%
\CSLLeftMargin{16. }%
\CSLRightInline{Luo W, Wang F. Measures of spatial accessibility to
health care in a {GIS} environment: Synthesis and a case study in the
chicago region. Environ Plann B Plann Des. 2003;30: 865--884.
doi:\href{https://doi.org/10.1068/b29120}{10.1068/b29120}}

\leavevmode\vadjust pre{\hypertarget{ref-shenLocationCharacteristicsInnercity1998}{}}%
\CSLLeftMargin{17. }%
\CSLRightInline{Shen Q. Location characteristics of inner-city
neighborhoods and employment accessibility of low-wage workers. Environ
Plann B. 1998;25: 345--365.
doi:\href{https://doi.org/10.1068/b250345}{10.1068/b250345}}

\leavevmode\vadjust pre{\hypertarget{ref-weibullAxiomaticApproachMeasurement1976}{}}%
\CSLLeftMargin{18. }%
\CSLRightInline{Weibull JW. An axiomatic approach to the measurement of
accessibility. Regional Science and Urban Economics. 1976;6: 357--379.
doi:\href{https://doi.org/10.1016/0166-0462(76)90031-4}{10.1016/0166-0462(76)90031-4}}

\leavevmode\vadjust pre{\hypertarget{ref-josephMeasuringPotentialPhysical1982}{}}%
\CSLLeftMargin{19. }%
\CSLRightInline{Joseph AE, Bantock PR. Measuring potential physical
accessibility to general practitioners in rural areas: A method and case
study. Social Science \& Medicine. 1982;16: 85--90.
doi:\href{https://doi.org/10.1016/0277-9536(82)90428-2}{10.1016/0277-9536(82)90428-2}}

\leavevmode\vadjust pre{\hypertarget{ref-taoInvestigatingImpactsPublic2020a}{}}%
\CSLLeftMargin{20. }%
\CSLRightInline{Tao Z, Zhou J, Lin X, Chao H, Li G. Investigating the
impacts of public transport on job accessibility in {Shenzhen}, {China}:
A multi-modal approach. Land Use Policy. 2020;99: 105025.
doi:\href{https://doi.org/10.1016/j.landusepol.2020.105025}{10.1016/j.landusepol.2020.105025}}

\leavevmode\vadjust pre{\hypertarget{ref-carpentieriMultimodalAccessibilityPrimary2020}{}}%
\CSLLeftMargin{21. }%
\CSLRightInline{Carpentieri G, Guida C, Masoumi HE. Multimodal
{Accessibility} to {Primary Health Services} for the {Elderly}: {A Case
Study} of {Naples}, {Italy}. Sustainability. 2020;12: 781.
doi:\href{https://doi.org/10.3390/su12030781}{10.3390/su12030781}}

\leavevmode\vadjust pre{\hypertarget{ref-espanaPlanNacionalIntegrado2020}{}}%
\CSLLeftMargin{22. }%
\CSLRightInline{España. Plan nacional integrado de energía y clima
({PNIEC}) 2021-2030. 2020 {[}cited 30 Jul 2023{]}. Available:
\url{https://www.miteco.gob.es/es/prensa/pniec.html}}

\leavevmode\vadjust pre{\hypertarget{ref-espanaResolucion10Enero2020}{}}%
\CSLLeftMargin{23. }%
\CSLRightInline{España. Resolución de 10 de enero de 2020, de la
dirección general de biodiversidad y calidad ambiental, por la que se
publica el programa nacional de control de la contaminación atmosférica.
Resolución Jan 24, 2020 pp. 6947--6947. Available:
\url{https://www.boe.es/eli/es/res/2020/01/10/(10)}}

\leavevmode\vadjust pre{\hypertarget{ref-barcelonaGUIATECNICAPARA2021}{}}%
\CSLLeftMargin{24. }%
\CSLRightInline{Barcelona. {GUÍA} {TÉCNICA} {PARA} {LA} {IMPLEMENTACIÓN}
{DE} {ZONAS} {DE} {BAJAS} {EMISIONES}. Àrea Metropolitana de Barcelona
({AMB}): Oficina Técnica de Gerencia del {AMB}; 2021. Available:
\url{https://revista.dgt.es/images/GUIA-ZBE.pdf}}

\leavevmode\vadjust pre{\hypertarget{ref-tarrinoortizAnalyzingImpactLow2022}{}}%
\CSLLeftMargin{25. }%
\CSLRightInline{Tarriño-Ortiz J, Gómez J, Soria-Lara JA, Vassallo JM.
Analyzing the impact of low emission zones on modal shift. Sustainable
Cities and Society. 2022;77: 103562.
doi:\href{https://doi.org/10.1016/j.scs.2021.103562}{10.1016/j.scs.2021.103562}}

\leavevmode\vadjust pre{\hypertarget{ref-comunidaddemadridResultadosEDM20182020}{}}%
\CSLLeftMargin{26. }%
\CSLRightInline{Comunidad de Madrid. Resultados de la {EDM} 2018 - Datos
Abiertos. 2020 {[}cited 31 Jul 2023{]}. Available:
\url{https://datos.comunidad.madrid/catalogo/dataset/resultados-edm2018}}

\leavevmode\vadjust pre{\hypertarget{ref-lopez_2017_spatial}{}}%
\CSLLeftMargin{27. }%
\CSLRightInline{Lopez FA, Paez A. Spatial clustering of high-tech
manufacturing and knowledge-intensive service firms in the greater
toronto area. Canadian Geographer-Geographe Canadien. 2017;61: 240--252.
doi:\href{https://doi.org/10.1111/cag.12326}{10.1111/cag.12326}}

\leavevmode\vadjust pre{\hypertarget{ref-horbachov_theoretical_2018}{}}%
\CSLLeftMargin{28. }%
\CSLRightInline{Horbachov P, Svichynskyi S. Theoretical substantiation
of trip length distribution for home-based work trips in urban transit
systems. Journal of Transport and Land Use. 2018;11: 593--632.
Available: \url{https://www.jstor.org/stable/26622420}}

\leavevmode\vadjust pre{\hypertarget{ref-batista_estimation_2019}{}}%
\CSLLeftMargin{29. }%
\CSLRightInline{Batista SFA, Leclercq L, Geroliminis N. Estimation of
regional trip length distributions for the calibration of the aggregated
network traffic models. Transportation Research Part B: Methodological.
2019;122: 192--217.
doi:\href{https://doi.org/10.1016/j.trb.2019.02.009}{10.1016/j.trb.2019.02.009}}

\leavevmode\vadjust pre{\hypertarget{ref-fitdistrplus_2015}{}}%
\CSLLeftMargin{30. }%
\CSLRightInline{Delignette-Muller ML, Dutang C. {fitdistrplus}: An {R}
package for fitting distributions. Journal of Statistical Software.
2015;64: 1--34. Available:
\url{https://www.jstatsoft.org/article/view/v064i04}}

\leavevmode\vadjust pre{\hypertarget{ref-reggianiAccessibilityImpedanceForms2011}{}}%
\CSLLeftMargin{31. }%
\CSLRightInline{Reggiani A, Bucci P, Russo G. Accessibility and
impedance forms: Empirical applications to the german commuting network.
International Regional Science Review. 2011;34: 230--252.
doi:\href{https://doi.org/10.1177/0160017610387296}{10.1177/0160017610387296}}

\leavevmode\vadjust pre{\hypertarget{ref-soukhovTTS2016RDataSet2023}{}}%
\CSLLeftMargin{32. }%
\CSLRightInline{Soukhov A, Páez A. {TTS}2016R: A data set to study
population and employment patterns from the 2016 transportation tomorrow
survey in the greater golden horseshoe area, ontario, canada.
Environment and Planning B: Urban Analytics and City Science. 2023;
23998083221146781.
doi:\href{https://doi.org/10.1177/23998083221146781}{10.1177/23998083221146781}}

\end{CSLReferences}

\nolinenumbers



\end{document}
